\cvsection{Projects}

\begin{cventries}

 \cventry
    {Flipkart GRiD 2.0}
    {Fashion Intelligence Systems}
    {\fontfamily{qcr}\selectfont \href{https://github.com/ShivenTripathi/GRiD-Software}{ShivenTripathi/GRiD-Software}}
    {July 2020 - Sep. 2020}
    {
      Implemented Machine Learning based model to identify trends in Fashion items and generate actionable insights on trendiness of products to build a portfolio.
      \begin{cvitems}
        
        % \item {Created a timeseries dataset from images scraped from ecommerce and fashion sites.}
        \item {Used open sourced MMFashion models to identify attributes and get features representative of styles.}
        \item {Built Fashion Vectors using a priority list obtained from the modified output MMFashion model to get individual trend ranks.}
        \item {Applied dimensionality reduction techniques like tSNE and then did clustering using kMeans to obtain visually distinct style clusters.}
        % \item {Integrated an XGBoost Regressor to get next timestep predictions for attributes future popularity.}
      
      \end{cvitems}
      Finished in the Top 3 teams over India under the problem statement of Fashion Intelligence Systems from over 9200 teams.
    }\\
   \cventry
    {Flipkart GRiD 2.0}
    {Autonomous Indoor Drone}
    {}
    {July 2020 - August 2020}
    {
      \begin{cvitems}
        \item {Studied and experimented various techniques related to 3D mapping of environment and maneuvring an autonomous drone.}
        \item {Evaluated approaches for building and completed simulations in a Gazebo environment.}
        \item {Employed colour segmentation and contour detection to identify gates, using Visual SLAM methods for mapping and navigation.}
      \end{cvitems}
      Finished amongst the Top 50 teams over the country to reach the Round 3 Qualifiers.
    }\\

  \cventry
    {Robotics CLub, IIT Kanpur}
    {Conversational Robot}
    {\fontfamily{qcr}\selectfont \href{https://github.com/ShivenTripathi/ConversationalRobot}{ShivenTripathi/ConversationalRobot}}
    {May 2020 - July 2020}
    {
      Developed a "talking bot" capable of listening to user, identifying intent and speaking out a meaningful response.
      \begin{cvitems}
        \item {Studied various preliminaries for Natural Language Processing and implemented GloVe model for generating embeddings for text.}
        \item {Experimented with different ASR methods and implemented DeepSpeech 2 trained on a limited but augmented dataset consisting of audiobooks.}
        \item {Built hybrid pipelines for response generation using pattern recognising AIML along with Topic Aware Seq2Seq Deep Learning model.}
      \end{cvitems}
    }

    \cventry
    {Robotics CLub, IIT Kanpur}
    {PETCat, Biomimetics}
    {\fontfamily{qcr}\selectfont \href{https://github.com/ShivenTripathi/PETcat_vision}{ShivenTripathi/PETcat\_vision}}
    {May 2020 - Present}
    {
      \begin{cvitems}
        \item {Working in the Vision subteam of the PETCat project which aims to build Biomimetic Robots.}
        \item {Implemented Face and Emotion Recognition modules using classic CV approaches combined with CNN architectures.}
        \item {Integrated individual modules into ROS nodes for easy deployment and functionality for the robot.}
      \end{cvitems}
    }

\end{cventries}
\vspace{-2mm}